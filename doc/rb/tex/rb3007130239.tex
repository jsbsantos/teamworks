\documentclass[]{article}
\usepackage{amssymb,amsmath}
\usepackage{ifxetex,ifluatex}
%\usepackage[toc,page]{appendix}
\ifxetex
\usepackage{fontspec,xltxtra,xunicode}
\defaultfontfeatures{Mapping=tex-text,Scale=MatchLowercase}
\newcommand{\euro}{€}
\else
\ifluatex
 \usepackage{fontspec}
 \defaultfontfeatures{Mapping=tex-text,Scale=MatchLowercase}
 \newcommand{\euro}{€}
\else
 \usepackage[utf8]{inputenc}
\fi
\fi
\usepackage{graphicx}
% We will generate all images so they have a width \maxwidth. This means
% that they will get their normal width if they fit onto the page, but
% are scaled down if they would overflow the margins.
\makeatletter
\makeatother
\ifxetex
\usepackage[setpagesize=false, % page size defined by xetex
           unicode=false, % unicode breaks when used with xetex
           xetex,
           bookmarks=true,
           pdfauthor={},
           pdftitle={},
           colorlinks=true,
           linkcolor=blue]{hyperref}
\else
\usepackage[unicode=true,
           bookmarks=true,
           pdfauthor={},
           pdftitle={},
           colorlinks=true,
           linkcolor=blue]{hyperref}
\fi
\hypersetup{breaklinks=true, pdfborder={0 0 0}}
\setlength{\parindent}{0pt}
\setlength{\parskip}{6pt plus 2pt minus 1pt}
\setlength{\emergencystretch}{3em}  % prevent overfull lines
\setcounter{secnumdepth}{0}
\renewcommand{\figurename}{Figura}
\renewcommand{\contentsname}{Indíce}
\renewcommand{\tablename}{Tabela}
\usepackage{placeins}
\begin{document}
\begin{titlepage}
\begin{center}
% Upper part of the page
% \includegraphics[width=0.15\textwidth]{./logo}\\[1cm] 
\textsc{\large Instituto Superior de Engenharia de Lisboa}\\[0.4cm]
\textsc{\large Área Departamental de Engenharia de Electrónica e Telecomunicações e de Computadores }\\[3cm]
% Title
{ \Huge \bfseries Teamworks}\\[1cm]
{ \Large \emph{a solution for project planning, managing and colaboration} }\\[1cm]
\textsc{\large Projecto e Seminário}\\[2cm]
\begin{minipage}{1\textwidth}
\begin{flushleft} \large
\textbf{\emph{Alunos}}\\
João Santos 30071 (a30071@alunos.isel.pt) \\
Filipe Pinheiro 30239 (a30239@alunos.isel.pt)
\end{flushleft}
\end{minipage}\\[0.5cm]
\begin{minipage}{1\textwidth}
\begin{flushright} \large
\textbf{\emph{Orientador}}\\
João Pedro Patriarca (jpatri@cc.isel.ipl.pt)
\end{flushright}
\end{minipage}
\vfill
% Bottom of the page
\begin{minipage}{1\textwidth}
\flushleft{Relatório beta do projecto Teamworks, desenvolvido no âmbito da unidade curricular de Projecto e Seminário do ano lectivo 2011/2012. Realizado pelos alunos Filipe Pinheiro e João Santos e orientado pelo docente João Pedro Patriarca.}\\[1cm]
\end{minipage}
\vfill
{\large Junho 2012}
\end{center}
\end{titlepage}
\tableofcontents

% include chapters
\end{document}


